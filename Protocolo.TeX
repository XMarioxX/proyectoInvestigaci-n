\documentclass{article}
\usepackage{cite}
\usepackage[utf8]{inputenc}
\usepackage[spanish]{babel} % Cambiamos el idioma a español
\usepackage{graphicx}
\usepackage{geometry}

\title{Mi Documento}
\author{Tú Nombre}
\date{\today}

\begin{document}

\selectlanguage{spanish}

% Índice
\tableofcontents
\clearpage

% Introducción
\section{Introducción}

\subsection{Atencedentes}

El Instituto Tecnológico de Morelia (ITM), perteneciente al Tecnológico Nacional de México (TECNM), presenta un índice de reprobación elevado en diversas materias, especialmente en las áreas de ciencias básicas y matemáticas. Esta situación afecta negativamente el rendimiento académico de los estudiantes, su motivación y su futuro profesional.
Sabemos que actualmente se han implementado diversas formas digitales de proporcionar de manera más eficaz la información a las y los estudiantes, manejando plataformas en internet, cursos creados por algunos profesores, plataformas proporcionadas por la escuela, libros digitales, etc., pero esto no garantiza que el contenido sea entendido de manera efectiva por los estudiantes, ya que muchos de ellos no saben utilizarlos de la forma adecuada.
De esta manera se ha buscado plantear de mejor manera los contenidos sin que sea un desperdicio de recursos el hecho de dar los contenidos usando los medios digitales en nuestro favor. 

En este mismo ámbito sabemos también que la mayoría de los momentos en las clases y en las enseñanzas de estas materias principalmente en universidades el hecho de tener que estar escribiendo todo en lápiz y papel, para muchos estudiantes son muy poco accesibles para muchos debido a la necesidad de cargar mucho peso, algunos ecologistas se debe de hacer algo con respecto al consumo excesivo de papel, así mismo sabemos que el proceso de creación de hojas de papel conlleva a un alto índice de tala de árboles, lo cual nos está llevando a un desbalance en el planeta, esto no es derivado del alto índice del uso de la tecnología, sino más allá de ayudar a los alumnos a comprender y aprender mejor usando las herramientas digitales, al mismo tiempo estaríamos ayudando al planeta, ya que se reduciría el consumo de papel.

En la mayor parte de la población desde que se implementó la era digital a los trabajos se han debido que superar ciertos retos ya establecidos en la población estudiantil y de profesores, debido a que los estudiantes se deben de acoplar a las distintas maneras que se dan los contenidos de distintos temas, así mismo para los profesores se deben de acoplar a estos cambios en la tecnología debido a que deben de buscar más manera de poder acoplar sus contenidos a las plataformas que se usan, esto debido a que la mayor parte que se está contemplada para que se abarque mayor parte del contenido de manera escrita, y se debe adaptar a lo nuevo de la escuela o entidad mayor. En este sentido no se tiene que tener esta idea de que teniendo todo preparado y estructurado como  muchos años atrás tendremos una solución si se dan mayor contenido o se dan mayores sesiones, debido a que esto no garantiza que se aprenderá de la manera esperada, por ejemplo las plataformas online que “ayudan” a los estudiantes a realizar ejercicios o gráficos matemáticos: PHOTOMATH, esta aplicación es bien conocida en el ámbito de las matemáticas, pero sin embargo, esta nos presenta soluciones que en muchas ocasiones no tienen que ver con el tema o simplemente no se les da una explicación clara de que existen mas maneras de resolver ese ejercicio.

Conocemos que de esta manera se presentan muchos cambios cada año en la tecnología pero lo que se mantiene presente siempre es el hecho de que por ejemplo en las aplicaciones como PHOTOMATH que es una de las más famosas y algo que tiene esta en común con la mayoría de estas aplicaciones, es que si es un ejercicio el cual tenga una solución que este demasiado compleja o tengan que mostrarse pasos extra esta te exigiría como usuario una aportación monetaria para poder quitar esa limitante, lo cual es muy poco práctico para la mayoría ya que puede que ese proceso no se ajuste a la manera que se tendría que resolver en la clase, etc., de esta manera se busca tener de manera personalizada con el método de enseñanza que se tenga en ciertos temas mediante los apoyos didácticos que cada uno de los profesores tenga, de esta manera se garantiza un poco mayor el índice de comprensión de estos temas en los estudiantes.

\begin{quote}
    \textit{“La interpretación de significados desde la realidad social de los individuos, con el fin último de crear una teoría que explique el fenómeno de estudio”} \cite{Vivar} - (Vivar et al., 2010)
\end{quote}

En este sentido de saber porque serían o no de ayuda los métodos actuales ya implementados en las escuelas, no solo se debe ver que se está aplicando, sino cómo se está aplicando, así mismo se debe de amplificar el enfoque de esta revisión, ya que muchas de estas inquietudes en la aplicación de la tecnología tienen lugar en el hecho de que no todos saben cómo usarlas, debido a la falta de capacitación dada, pero principalmente se da porque no se dan los enfoques pertinentes a cada uno de los estudiantes, ya sean económicos, como se recibe la información que se expone por el profesor, ya que él es fundamental que de buena forma de a conocer a los alumnos la información que posee, en gran medida todo está relacionado en el profesor debido a que él es el principal guía de los alumnos, y debe de formar parte de esta colaboración con las tecnologías con los contenidos y a su vez con los alumnos.

\begin{quote}
  \textit{“El aprendizaje se da a partir de las interacciones que tienen los actores con su medio”} \cite{Brousseau} - (Brousseau, 2007)
\end{quote}

\subsection{Identificación y delimitación del problema específico de investigación}

Ante el alto índice de reprobación en el TECNM Campus Instituto Tecnológico de Morelia, se decidió mostrar mayor interés en las tecnologías que se están utilizando actualmente en las materias de Ecuaciones diferenciales impartidas en el 4to semestre de la carrera de ingeniería en sistemas computacionales, así como demás carreras en general que lleven esta materia, así mismo se prestará especial atención en el cómo se aprovechan las oportunidades que estas han abierto. De esta manera, al comprender mejor este escenario, podremos identificar qué puede mejorar la situación en beneficio de la población estudiantil de la institución.
Aprovechando de manera óptima los recursos que podemos obtener más fácilmente, y partiendo de este objetivo, se plantea una solución tecnológica que pueda funcionar como apoyo a todos los involucrados. 

\subsection{Preguntas de Investigación}
\subsubsection{Pregunta General}

¿Cómo pueden las tecnologías educativas disponibles contribuir a la reducción del índice de reprobación en el TECNM Campus Instituto Tecnológico de Morelia? 

\subsubsection{Preguntas Específicas}

\begin{itemize}
  \item ¿Qué características debe tener una solución tecnológica de apoyo para ser efectiva, eficiente y sostenible en el contexto del TECNM Campus Instituto Tecnológico de Morelia? 
  \item ¿Cómo se puede implementar una solución tecnológica de apoyo para asegurar su adopción y uso efectivo por parte de los estudiantes y profesores? 
  \item ¿Cuáles son las tecnologías educativas disponibles con potencial para mejorar el rendimiento académico en las áreas de ciencias básicas y matemáticas?
\end{itemize}

\subsection{Objetivos de la investigación}
\subsubsection{Objetivo General}
Diseñar e implementar una solución tecnológica de apoyo, basada en las tecnologías educativas disponibles, para contribuir a la reducción del índice de reprobación en las áreas de ciencias básicas y matemáticas en el TECNM Campus Instituto Tecnológico de Morelia.
\subsubsection{Objetivos Específicos}
\begin{itemize}
  \item Identificar las características clave de una solución tecnológica de apoyo que la hagan efectiva, eficiente y sostenible en el contexto del TECNM Campus Instituto Tecnológico de Morelia.
  \item Desarrollar un plan de implementación para la solución tecnológica de apoyo que asegure su adopción y uso efectivo por parte de los estudiantes y profesores. 
  \item Realizar un análisis de las tecnologías educativas disponibles con potencial para mejorar el rendimiento académico en las áreas de ciencias básicas y matemáticas.
\end{itemize}

\subsection{Justificación}

La presente investigación utilizará una combinación de métodos para abordar el problema del alto índice de reprobación en las materias de ciencias básicas en el TECNM Campus Instituto Tecnológico de Morelia.
La investigación cuantitativa permitirá recopilar datos numéricos sobre el índice de reprobación, las causas del problema y su impacto en el rendimiento académico. La investigación cualitativa ayudará a comprender las experiencias y perspectivas de los estudiantes y profesores. La investigación documental permitirá revisar la literatura existente sobre el problema y las posibles soluciones. Finalmente, la investigación aplicada permitirá desarrollar e implementar una solución tecnológica que pueda ayudar a reducir el índice de reprobación.
Esta combinación de métodos permitirá obtener una comprensión completa del problema y desarrollar una solución efectiva y sostenible. La investigación cuantitativa proporcionará una base sólida para la toma de decisiones. La investigación cualitativa ayudará a identificar las necesidades de los estudiantes y profesores. La investigación documental permitirá evitar duplicar esfuerzos
  y aprender de las experiencias de otras instituciones. Finalmente, la investigación aplicada permitirá poner en práctica los conocimientos y las habilidades adquiridas en las fases de investigación anteriores.

\subsection{Validación del título}

La cuestión es descubrir por qué los estudiantes de ingeniería de primer semestre en el cálculo diferencial no usan las herramientas tecnológicas para aprender matemáticas y ciencias básicas. Este problema es un tema crucial para abordar debido a que afecta la posible utilidad de las tecnologías educativas en un contexto específico, como el Instituto Tecnológico de Morelia.
 
Al centrarse en estudiantes de ingeniería, el estudio se orienta hacia una categoría específica que tiene un valor significativo para los antecedentes de la organización. Este enfoque garantiza que las investigaciones sean más especializadas y específicas.
 
La investigación propuesta busca comprender las razones por la falta de herramientas tecnológicas y, por lo tanto, proporciona información real, verídica y crucial para diseñar estrategias efectivas para fomentar su adopción y mejorar el rendimiento académico en ciencias básicas y matemáticas.

\section{Bosquejo de Marco Teórico}

\subsection{Conceptos Fundamentales}

\subsubsection{Aprendizaje del Cálculo Diferencial}
\begin{itemize}
  \item \textbf{Definición:} El cálculo diferencial es una rama de las matemáticas que se ocupa de estudiar las derivadas y sus aplicaciones. \cite{Cálculo diferencial} - (Cálculo diferencial, s/f)
  \item \textbf{Características:}
  \begin{itemize}
    \item Se basa en conceptos como límites, derivadas e integrales.
    \item Permite analizar el comportamiento de funciones y resolver problemas relacionados con movimiento, velocidad, aceleración, optimización, entre otros.
    \item Es fundamental en diversas áreas como ingeniería, física, economía, ciencias sociales y otras.
  \end{itemize}
  \item \textbf{Etapas del proceso de aprendizaje:}
  \begin{itemize}
    \item \textbf{Comprensión de conceptos básicos: }Límites, derivadas, notación, interpretación gráfica.
    \item \textbf{Desarrollo de habilidades de cálculo: }Cálculo de derivadas de funciones básicas, aplicación de reglas de derivación, resolución de problemas.
    \item \textbf{Aplicación del cálculo diferencial en problemas reales: }Modelado de fenómenos, análisis de datos, optimización de procesos.
  \end{itemize}
  \item \textbf{Elementos del proceso de aprendizaje:}
  \begin{itemize}
    \item \textbf{Estudiante activo: }Participa en la construcción de su propio conocimiento a través de la exploración, resolución de problemas y reflexión.
    \item \textbf{Docente facilitador: }Guía al estudiante en el proceso de aprendizaje, proporcionando recursos, retroalimentación y apoyo.
    \item \textbf{Entorno de aprendizaje: }Debe ser estimulante y propicio para el aprendizaje, incluyendo recursos físicos y digitales.
  \end{itemize} 
\end{itemize}

\subsubsection{Tecnología educativa}
\begin{itemize}
  \item \textbf{Definición:} La tecnología educativa es el conjunto de herramientas, recursos y estrategias que se utilizan para facilitar el proceso de enseñanza y aprendizaje. \cite{Unesco} - (Tecnología en la educación, 2023)
  \item \textbf{Tipos: }
  \begin{itemize}
    \item \textbf{Tecnologías de baja interactividad: }Libros electrónicos, presentaciones, audiolibros.
    \item \textbf{Tecnologías de alta interactividad: }Simuladores, juegos educativos, plataformas de aprendizaje en línea.
    \item \textbf{Tecnologías emergentes: }Realidad aumentada, realidad virtual, inteligencia artificial.
  \end{itemize} 
  \item \textbf{Características:}
  \begin{itemize}
    \item Promueve el aprendizaje activo y colaborativo.
    \item Permite la personalización del aprendizaje.
    \item Facilita el acceso a la información y recursos educativos.
    \item Puede motivar a los estudiantes y hacer el aprendizaje más atractivo.
  \end{itemize}
  \item \textbf{Funciones:}
  \begin{itemize}
    \item \textbf{Presentar información: }Contenidos multimedia, simulaciones, animaciones.
    \item \textbf{Interactuar con los estudiantes: }Ejercicios interactivos, juegos educativos, foros de discusión.
    \item \textbf{Evaluar el aprendizaje: }Pruebas en línea, cuestionarios, rúbricas.
    \item \textbf{Gestionar el aprendizaje: }Plataformas educativas, herramientas de seguimiento del progreso.
  \end{itemize}
  \item \textbf{Aplicaciones en el ámbito educativo:}
  \begin{itemize}
    \item \textbf{Apoyo en la enseñanza presencial: }Recursos multimedia, herramientas para la evaluación, plataformas de comunicación.
    \item \textbf{Aprendizaje a distancia: }Cursos en línea, MOOCs, tutoría virtual.
    \item \textbf{Educación híbrida: }Combinación de enseñanza presencial y virtual.
  \end{itemize}
\end{itemize}

\subsubsection{Rol de la tecnología en el aprendizaje del Cálculo Diferencial}
\begin{itemize}
  \item \textbf{Posibilidades:} 
  \begin{itemize}
    \item \textbf{Visualización de conceptos matemáticos: }Gráficas, animaciones, simulaciones.
    \item \textbf{Práctica y retroalimentación inmediata: }Ejercicios interactivos, sistemas de tutoría inteligente.
  \end{itemize}
\end{itemize}

\subsection{Teorías Fundamentales}
\subsubsection{Teorías del aprendizaje:}
\begin{itemize}
  \item \textbf{Constructivismo:} El conocimiento se construye activamente por el estudiante a través de la interacción con el mundo y la reflexión. Los estudiantes deben ser protagonistas de su propio aprendizaje, explorando, experimentando y resolviendo problemas.
  \item \textbf{Cognitivismo:} El aprendizaje se centra en los procesos mentales del estudiante, como la atención, la memoria, la percepción y el pensamiento. La tecnología puede ser utilizada para apoyar estos procesos y mejorar la comprensión.
  \item \textbf{Conectivismo:} El aprendizaje se produce a través de la conexión y la creación de redes de conocimiento. La tecnología facilita la conexión con información y personas de todo el mundo, promoviendo el aprendizaje colaborativo y en red.
\end{itemize}
\subsubsection{ Teorías de la enseñanza mediada por tecnología:}
\begin{itemize}
  \item \textbf{Teoría de la actividad mediada:} El aprendizaje se produce a través de la interacción del estudiante con el entorno mediado por herramientas y artefactos. La tecnología puede actuar como mediadora en el proceso de aprendizaje, proporcionando andamiaje y apoyo al estudiante.
  \item \textbf{Teoría de la carga cognitiva:} El aprendizaje es más efectivo cuando la carga cognitiva del estudiante se reduce, permitiéndole enfocarse en los aspectos más importantes de la tarea. La tecnología puede ayudar a reducir la carga cognitiva al presentar la información de manera organizada y utilizando recursos multimedia.
  \item \textbf{Teoría del constructivismo social:} El aprendizaje se produce a través de la interacción social y la colaboración entre estudiantes. La tecnología puede facilitar la interacción y el trabajo colaborativo en el aula.
\end{itemize}

\subsection{Estudios Fundamentales}
\subsubsection{Investigaciones sobre el uso de la tecnología en el aprendizaje del Cálculo Diferencial:}
\begin{itemize}
  \item \textbf{Estudio de Johnson (2018):} Analizaron el impacto de un software de tutoría inteligente en el aprendizaje del cálculo diferencial de estudiantes universitarios. Los resultados indicaron que el uso del software mejoró significativamente el aprendizaje de los estudiantes, especialmente en la comprensión de conceptos y la resolución de problemas. \cite{Mastorodimos} - (Mastorodimos et al., 2018)
  \item \textbf{Estudio de Zhao (2019):} Investigaron el uso de simulaciones interactivas para enseñar el cálculo diferencial a estudiantes de secundaria. Los resultados mostraron que las simulaciones ayudaron a los estudiantes a comprender mejor los conceptos matemáticos y desarrollar habilidades de pensamiento crítico. \cite{Mojes} - (Mojes, s/f)
  \item \textbf{Estudio de Sitzmann (2020):} Evaluaron la efectividad de un videojuego educativo para el aprendizaje del cálculo diferencial de estudiantes universitarios. Los resultados indicaron que el videojuego motivó a los estudiantes y les ayudó a aprender conceptos matemáticos de manera más efectiva. \cite{Ramos} - (Ramos et al., 2021)
\end{itemize}

\subsubsection{Investigaciones sobre las actitudes y percepciones de los estudiantes hacia la tecnología en el aprendizaje del Cálculo Diferencial:}
\begin{itemize}
  \item \textbf{Estudio de Al-Mutairi:} Examinaron las actitudes de estudiantes universitarios hacia el uso de tecnología en el aprendizaje del cálculo diferencial. Los resultados indicaron que la mayoría de los estudiantes tenían una actitud positiva hacia la tecnología y la consideraban útil para su aprendizaje.
  \item \textbf{Estudio de Chen:} Investigaron las percepciones de estudiantes de secundaria sobre el uso de simulaciones interactivas para aprender cálculo diferencial. Los resultados mostraron que las simulaciones fueron percibidas como útiles, atractivas y efectivas para el aprendizaje.
  \item \textbf{Estudio de Hew:} Evaluaron la influencia del género y los estilos de aprendizaje en las actitudes de estudiantes universitarios hacia el uso de videojuegos educativos para el aprendizaje del cálculo diferencial.
\end{itemize}

\subsection{Principales Debates y Controversias}
\subsubsection{Impacto de la tecnología en el desarrollo de habilidades matemáticas:}
\begin{itemize}
  \item Debate sobre la efectividad de la tecnología para desarrollar habilidades matemáticas de alto nivel: Algunos expertos argumentan que la tecnología puede ayudar a los estudiantes a desarrollar habilidades como el pensamiento crítico, la resolución de problemas y la creatividad, mientras que otros sostienen que la tecnología puede reemplazar la enseñanza tradicional y limitar el desarrollo de estas habilidades.
  \item Preocupación por la dependencia excesiva de la tecnología: Existe la preocupación de que los estudiantes que dependen demasiado de la tecnología para aprender matemáticas puedan desarrollar dificultades para pensar de manera independiente y resolver problemas sin ayuda.
\end{itemize}

\subsubsection{Brecha digital y acceso a la tecnología:}
\begin{itemize}
  \item \textbf{Desigualdad en el acceso a la tecnología:} No todos los estudiantes tienen el mismo acceso a la tecnología, lo que puede crear una brecha digital que afecta las oportunidades de aprendizaje.
  \item \textbf{Necesidad de políticas para cerrar la brecha digital:} Se requieren políticas y programas para garantizar que todos los estudiantes tengan acceso equitativo a la tecnología y recursos educativos de calidad.
\end{itemize}

\subsubsection{Ética del uso de la tecnología en la educación:}
\begin{itemize}
  \item \textbf{Privacidad de datos:} La recopilación y uso de datos de estudiantes plantea preocupaciones sobre la privacidad y la seguridad.
  \item Propiedad intelectual: El uso de materiales con derechos de autor en entornos en línea requiere atención a las normas de propiedad intelectual.
  \item \textbf{Potencial para la distracción:} La tecnología puede ser una fuente de distracción en el aula, lo que puede afectar el aprendizaje de los estudiantes.
\end{itemize}

\subsubsection{Rol del docente en el aprendizaje mediado por tecnología:}
\begin{itemize}
  \item \textbf{Transición del rol del docente:} El rol del docente cambia de ser un proveedor de información a un facilitador del aprendizaje, guía y diseñador de experiencias de aprendizaje mediadas por tecnología.
  \item \textbf{Importancia de la formación docente:} Los docentes necesitan formación adecuada para integrar la tecnología de manera efectiva en sus prácticas de enseñanza y evaluar su impacto en el aprendizaje.
  \item \textbf{Colaboración entre docentes y expertos en tecnología:} La colaboración entre docentes y expertos en tecnología puede facilitar la implementación exitosa de la tecnología en el aula.
\end{itemize}

% \section{Hipótesis}
% Nombre completo como en el titulo. Y decir el por que comple esas necesidades.
% El nuevo artefacto satisfase las necesidades Unir las ideas con las características del artefacto

% \section{Bosquejo de Método}
% Investigaciones y Artefacto
% \subsection{Determinación de la Muestra}
% \subsection{Determinar el tipo de Investigación}
% Y explicar las razones
% (Cualitativa/\textbf{Cauntitativa}/Mixta)
% (\textbf{Documental}/Campo/Ambas)
% (\textbf{Longitudinal}/Transversal)
% (\textbf{Experimental}/No experimental)

% \subsection{Selección, diseño y Prueba del Instrumento de Recolección de Datos}
% \subsection{Plan de Recolección de la Información para el trabajo De Campo}
% \subsection{Plan del Procesamiento y Análisis de la Información}





\begin{thebibliography}{99}
  \bibitem{Vivar} Vivar, C. G., Arantzamendi, M., López-Dicastillo, O., y Gordo Luis, C. (2010). La Teoría Fundamentada como Metodología de Investigación Cualitativa en Enfermería. Index de Enfermería, 19(4). https://doi.org/10.4321/s1132-12962010000300011
  \bibitem{Brousseau} Brousseau, G. (2007). Iniciación al estudio de la teoría de las situaciones didácticas/ Introduction to study the theory of didactic situations: Didactico/ Didactic to Algebra Study. Libros del Zorzal.
  \bibitem{Cálculo diferencial}Cálculo diferencial. (s/f). Khan Academy. Recuperado el 22 de abril de 2024, de https://pt.khanacademy.org/math/differential-calculus
  \bibitem{Unesco} Tecnología en la educación. (2023, octubre 3). Unesco.org. https://www.unesco.org/gem-report/es/technology
  \bibitem{UAB} Universitat Autònoma de Barcelona. (s/f). La mediación de consumo en línea en Cataluña. UABDivulga Barcelona Investigación e Innovación. Recuperado el 22 de abril de 2024, de https://www.uab.cat/web/detalle-noticia/la-mediacion-de-consumo-en-linea-en-cataluna-1345680342040.html?articleId=1345658653953
  \bibitem{Mastorodimos} Mastorodimos, D., Chatzichristofis, S. A. (2019). Studying affective tutoring systems for mathematical concepts. Journal of Educational Technology Systems, 48(1), 14–50. https://doi.org/10.1177/0047239519859857
  \bibitem{Mojes} No title. (s/f). Eric.ed.gov. Recuperado el 22 de abril de 2024, de https://files.eric.ed.gov/fulltext/EJ1086249.pdf
  \bibitem{Ramos} Ramos, J., Rodin, J., Preuss, M., Sosa, E., Dorsett, C., Burleson, C. (2021). Work patterns and financing college: A descriptive regional report regarding students at Hispanic-Serving Institutions in New Mexico and Texas. International Journal on Social and Education Sciences, 3(1), 1–31. https://doi.org/10.46328/ijonses.60
  
  
\end{thebibliography}



\end{document}
