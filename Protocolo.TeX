\documentclass{article}
\usepackage[utf8]{inputenc}
\usepackage[spanish]{babel} % Cambiamos el idioma a español
\usepackage{graphicx}
\usepackage{geometry}

\title{Mi Documento}
\author{Tú Nombre}
\date{\today}

\begin{document}

% Portada
\begin{titlepage}
    \centering
    \vspace*{5cm}
    {\Huge\textbf{\title}\par}
    \vspace{1cm}
    {\Large\textit{\author}\par}
    \vspace{1cm}
    {\large\textit{\date}\par}
    \vfill
    \includegraphics[width=0.3\textwidth]{example-image} 
    \vfill
\end{titlepage}

\selectlanguage{spanish}

% Índice
\tableofcontents
\clearpage

% Introducción
\section{Introducción}

\subsection{Planteamiento General del Problema de Investigación }


Existe cierta intriga tanto desde el punto de vista estudiantil como desde el de los profesores.
Es bien sabido que, desde hace mucho tiempo, en la mayoría de los primeros semestres del nivel superior,
se mantiene un alto índice reprobatorio en las materias consideradas más difíciles o que requieren mayor
 esfuerzo en cuanto a tiempo y dedicación. En este contexto, nos enfocamos en el alto índice de reprobación 
 en las materias de ciencias básicas impartidas en los primeros semestres (1°, 2°, 3°) en el TECNM Campus Instituto Tecnológico
  de Morelia. Dicho índice tiene un alto impacto en el rendimiento académico de los estudiantes. En este sentido, nos referimos
   específicamente a las medidas que se toman en estos casos, como el estímulo que se les da a los estudiantes y cómo lo aprovechan.

\subsection{Definición del Problema General de Investigación}

En el TECNM Campus Instituto Tecnológico de Morelia, existe un alto índice de reprobación en las materias de ciencias básicas de los
 primeros semestres, lo que afecta negativamente el rendimiento académico de los estudiantes. Se busca analizar las medidas de apoyo
  existentes y su aprovechamiento por parte de los estudiantes, con el fin de proponer estrategias para mejorar la situación.

\subsection{Identificación y delimitación de un problema}
Ante el alto índice de reprobación en el TECNM Campus Instituto Tecnológico de Morelia,
 se decidió mostrar mayor interés en las tecnologías que se están utilizando actualmente, así como en las
  oportunidades que estas han abierto. De esta manera, al comprender mejor este escenario, podremos identificar
   qué puede mejorar la situación en beneficio de la población estudiantil de la institución.

Aprovechando de manera óptima los recursos que podemos obtener más fácilmente, y partiendo de este objetivo, 
se plantea una solución tecnológica que pueda funcionar como apoyo a todos los involucrados. 

\subsection{Enunciado del problema de Investigación}

El TECNM Campus Instituto Tecnológico de Morelia presenta un alto índice de reprobación en las diferentes materias,
 especialmente en las áreas de ciencias básicas y matemáticas. Esta situación afecta negativamente el rendimiento académico
  de los estudiantes, su motivación y su futuro profesional.
  Diversos factores inciden en el alto índice de reprobación, como la dificultad de las materias, la falta de estrategias de
   aprendizaje adecuadas por parte de los estudiantes, las deficiencias en los métodos de enseñanza tradicionales y la poca
    disponibilidad de recursos educativos innovadores y accesibles.

Sin embargo, existen numerosas oportunidades de mejora. Las tecnologías educativas disponibles ofrecen nuevas posibilidades
 para mejorar el proceso de enseñanza-aprendizaje. Se pueden implementar estrategias de aprendizaje activo y personalizado, 
 crear recursos educativos interactivos y multimedia, y brindar apoyo y seguimiento a los estudiantes.

\subsection{Preguntas de Investigación}
    \subsubsection{Pregunta General}
    
    ¿Cómo pueden las tecnologías educativas disponibles contribuir a la reducción del índice de reprobación en el TECNM
     Campus Instituto Tecnológico de Morelia?

    \subsubsection{Preguntas Específicas}

    ¿Qué características debe tener una solución tecnológica de apoyo para ser efectiva, eficiente y sostenible en el
     contexto del TECNM Campus Instituto Tecnológico de Morelia? \\

     ¿Cómo se puede implementar una solución tecnológica de apoyo para asegurar su adopción y uso efectivo por parte de
      los estudiantes y profesores? \\

      ¿Cuáles son las tecnologías educativas disponibles con potencial para mejorar el rendimiento académico en las áreas
       de ciencias básicas y matemáticas?

\subsection{Objetivos}
    \subsubsection{Objetivo General}

    Diseñar e implementar una solución tecnológica de apoyo, basada en las tecnologías educativas disponibles, para
     contribuir a la reducción del índice de reprobación en las áreas de ciencias básicas y matemáticas en el TECNM
      Campus Instituto Tecnológico de Morelia.

    \subsubsection{Objetivos Específicos}
    \begin{itemize}
        \item Identificar las características clave de una solución tecnológica de apoyo que la hagan efectiva, eficiente y 
        sostenible en el contexto del TECNM Campus Instituto Tecnológico de Morelia.
    
        \item Desarrollar un plan de implementación para la solución tecnológica de apoyo que asegure su adopción y uso 
        efectivo por parte de los estudiantes y profesores.

        \item Realizar un análisis de las tecnologías educativas disponibles con potencial para mejorar el rendimiento
         académico en las áreas de ciencias básicas y matemáticas
        \end{itemize}

\subsection{Justificación}
La presente investigación utilizará una combinación de métodos para abordar el problema del alto índice de reprobación
 en las materias de ciencias básicas en el TECNM Campus Instituto Tecnológico de Morelia.

La investigación cuantitativa permitirá recopilar datos numéricos sobre el índice de reprobación, las causas del problema
 y su impacto en el rendimiento académico. La investigación cualitativa ayudará a comprender las experiencias y perspectivas
  de los estudiantes y profesores. La investigación documental permitirá revisar la literatura existente sobre el problema y 
  las posibles soluciones. Finalmente, la investigación aplicada permitirá desarrollar e implementar una solución tecnológica
   que pueda ayudar a reducir el índice de reprobación.

Esta combinación de métodos permitirá obtener una comprensión completa del problema y desarrollar una solución efectiva y 
sostenible. La investigación cuantitativa proporcionará una base sólida para la toma de decisiones. La investigación cualitativa
 ayudará a identificar las necesidades de los estudiantes y profesores. La investigación documental permitirá evitar duplicar esfuerzos
  y aprender de las experiencias de otras instituciones. Finalmente, la investigación aplicada permitirá poner en práctica los 
  conocimientos y las habilidades adquiridas en las fases de investigación anteriores.
  
\subsection{Validación del Título}

\end{document}
