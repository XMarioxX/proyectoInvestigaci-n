\documentclass{article}
\usepackage{cite}
\usepackage[utf8]{inputenc}
\usepackage[spanish]{babel} % Cambiamos el idioma a español
\usepackage{graphicx}
\usepackage{geometry}

\title{Mi Documento}
\author{Tú Nombre}
\date{\today}

\begin{document}

\selectlanguage{spanish}

% Índice
\tableofcontents
\clearpage

% Introducción
\section{Introducción}

\subsection{Atencedentes}

El Instituto Tecnológico de Morelia (ITM), perteneciente al Tecnológico Nacional de México (TECNM), presenta un índice de reprobación elevado en diversas materias, especialmente en las áreas de ciencias básicas y matemáticas. Esta situación afecta negativamente el rendimiento académico de los estudiantes, su motivación y su futuro profesional.
Sabemos que actualmente se han implementado diversas formas digitales de proporcionar de manera más eficaz la información a las y los estudiantes, manejando plataformas en internet, cursos creados por algunos profesores, plataformas proporcionadas por la escuela, libros digitales, etc., pero esto no garantiza que el contenido sea entendido de manera efectiva por los estudiantes, ya que muchos de ellos no saben utilizarlos de la forma adecuada.
De esta manera se ha buscado plantear de mejor manera los contenidos sin que sea un desperdicio de recursos el hecho de dar los contenidos usando los medios digitales en nuestro favor. 

En este mismo ámbito sabemos también que la mayoría de los momentos en las clases y en las enseñanzas de estas materias principalmente en universidades el hecho de tener que estar escribiendo todo en lápiz y papel, para muchos estudiantes son muy poco accesibles para muchos debido a la necesidad de cargar mucho peso, algunos ecologistas se debe de hacer algo con respecto al consumo excesivo de papel, así mismo sabemos que el proceso de creación de hojas de papel conlleva a un alto índice de tala de árboles, lo cual nos está llevando a un desbalance en el planeta, esto no es derivado del alto índice del uso de la tecnología, sino más allá de ayudar a los alumnos a comprender y aprender mejor usando las herramientas digitales, al mismo tiempo estaríamos ayudando al planeta, ya que se reduciría el consumo de papel.

En la mayor parte de la población desde que se implementó la era digital a los trabajos se han debido que superar ciertos retos ya establecidos en la población estudiantil y de profesores, debido a que los estudiantes se deben de acoplar a las distintas maneras que se dan los contenidos de distintos temas, así mismo para los profesores se deben de acoplar a estos cambios en la tecnología debido a que deben de buscar más manera de poder acoplar sus contenidos a las plataformas que se usan, esto debido a que la mayor parte que se está contemplada para que se abarque mayor parte del contenido de manera escrita, y se debe adaptar a lo nuevo de la escuela o entidad mayor. En este sentido no se tiene que tener esta idea de que teniendo todo preparado y estructurado como  muchos años atrás tendremos una solución si se dan mayor contenido o se dan mayores sesiones, debido a que esto no garantiza que se aprenderá de la manera esperada, por ejemplo las plataformas online que “ayudan” a los estudiantes a realizar ejercicios o gráficos matemáticos: PHOTOMATH, esta aplicación es bien conocida en el ámbito de las matemáticas, pero sin embargo, esta nos presenta soluciones que en muchas ocasiones no tienen que ver con el tema o simplemente no se les da una explicación clara de que existen mas maneras de resolver ese ejercicio.

Conocemos que de esta manera se presentan muchos cambios cada año en la tecnología pero lo que se mantiene presente siempre es el hecho de que por ejemplo en las aplicaciones como PHOTOMATH que es una de las más famosas y algo que tiene esta en común con la mayoría de estas aplicaciones, es que si es un ejercicio el cual tenga una solución que este demasiado compleja o tengan que mostrarse pasos extra esta te exigiría como usuario una aportación monetaria para poder quitar esa limitante, lo cual es muy poco práctico para la mayoría ya que puede que ese proceso no se ajuste a la manera que se tendría que resolver en la clase, etc., de esta manera se busca tener de manera personalizada con el método de enseñanza que se tenga en ciertos temas mediante los apoyos didácticos que cada uno de los profesores tenga, de esta manera se garantiza un poco mayor el índice de comprensión de estos temas en los estudiantes.

\begin{quote}
    \textit{“La interpretación de significados desde la realidad social de los individuos, con el fin último de crear una teoría que explique el fenómeno de estudio”} \cite{Vivar} - (Vivar et al., 2010)
\end{quote}

En este sentido de saber porque serían o no de ayuda los métodos actuales ya implementados en las escuelas, no solo se debe ver que se está aplicando, sino cómo se está aplicando, así mismo se debe de amplificar el enfoque de esta revisión, ya que muchas de estas inquietudes en la aplicación de la tecnología tienen lugar en el hecho de que no todos saben cómo usarlas, debido a la falta de capacitación dada, pero principalmente se da porque no se dan los enfoques pertinentes a cada uno de los estudiantes, ya sean económicos, como se recibe la información que se expone por el profesor, ya que él es fundamental que de buena forma de a conocer a los alumnos la información que posee, en gran medida todo está relacionado en el profesor debido a que él es el principal guía de los alumnos, y debe de formar parte de esta colaboración con las tecnologías con los contenidos y a su vez con los alumnos.

\begin{quote}
  \textit{“El aprendizaje se da a partir de las interacciones que tienen los actores con su medio”} \cite{Brousseau} - (Brousseau, 2007)
\end{quote}

\subsection{Identificación y delimitación del problema específico de investigación}

Ante el alto índice de reprobación en el TECNM Campus Instituto Tecnológico de Morelia, se decidió mostrar mayor interés en las tecnologías que se están utilizando actualmente en las materias de Ecuaciones diferenciales impartidas en el 4to semestre de la carrera de ingeniería en sistemas computacionales, así como demás carreras en general que lleven esta materia, así mismo se prestará especial atención en el cómo se aprovechan las oportunidades que estas han abierto. De esta manera, al comprender mejor este escenario, podremos identificar qué puede mejorar la situación en beneficio de la población estudiantil de la institución.
Aprovechando de manera óptima los recursos que podemos obtener más fácilmente, y partiendo de este objetivo, se plantea una solución tecnológica que pueda funcionar como apoyo a todos los involucrados. 

\subsection{Preguntas de Investigación}
\subsubsection{Pregunta General}

¿Cómo pueden las tecnologías educativas disponibles contribuir a la reducción del índice de reprobación en el TECNM Campus Instituto Tecnológico de Morelia? 

\subsubsection{Preguntas Específicas}

\begin{itemize}
  \item ¿Qué características debe tener una solución tecnológica de apoyo para ser efectiva, eficiente y sostenible en el contexto del TECNM Campus Instituto Tecnológico de Morelia? 
  \item ¿Cómo se puede implementar una solución tecnológica de apoyo para asegurar su adopción y uso efectivo por parte de los estudiantes y profesores? 
  \item ¿Cuáles son las tecnologías educativas disponibles con potencial para mejorar el rendimiento académico en las áreas de ciencias básicas y matemáticas?
\end{itemize}

\subsection{Objetivos de la investigación}
\subsubsection{Objetivo General}
Diseñar e implementar una solución tecnológica de apoyo, basada en las tecnologías educativas disponibles, para contribuir a la reducción del índice de reprobación en las áreas de ciencias básicas y matemáticas en el TECNM Campus Instituto Tecnológico de Morelia.
\subsubsection{Objetivos Específicos}
\begin{itemize}
  \item Identificar las características clave de una solución tecnológica de apoyo que la hagan efectiva, eficiente y sostenible en el contexto del TECNM Campus Instituto Tecnológico de Morelia.
  \item Desarrollar un plan de implementación para la solución tecnológica de apoyo que asegure su adopción y uso efectivo por parte de los estudiantes y profesores. 
  \item Realizar un análisis de las tecnologías educativas disponibles con potencial para mejorar el rendimiento académico en las áreas de ciencias básicas y matemáticas.
\end{itemize}

\subsection{Justificación}

La presente investigación utilizará una combinación de métodos para abordar el problema del alto índice de reprobación en las materias de ciencias básicas en el TECNM Campus Instituto Tecnológico de Morelia.
La investigación cuantitativa permitirá recopilar datos numéricos sobre el índice de reprobación, las causas del problema y su impacto en el rendimiento académico. La investigación cualitativa ayudará a comprender las experiencias y perspectivas de los estudiantes y profesores. La investigación documental permitirá revisar la literatura existente sobre el problema y las posibles soluciones. Finalmente, la investigación aplicada permitirá desarrollar e implementar una solución tecnológica que pueda ayudar a reducir el índice de reprobación.
Esta combinación de métodos permitirá obtener una comprensión completa del problema y desarrollar una solución efectiva y sostenible. La investigación cuantitativa proporcionará una base sólida para la toma de decisiones. La investigación cualitativa ayudará a identificar las necesidades de los estudiantes y profesores. La investigación documental permitirá evitar duplicar esfuerzos
  y aprender de las experiencias de otras instituciones. Finalmente, la investigación aplicada permitirá poner en práctica los conocimientos y las habilidades adquiridas en las fases de investigación anteriores.

\subsection{Validación del título}

La cuestión es descubrir por qué los estudiantes de ingeniería de primer semestre en el cálculo diferencial no usan las herramientas tecnológicas para aprender matemáticas y ciencias básicas. Este problema es un tema crucial para abordar debido a que afecta la posible utilidad de las tecnologías educativas en un contexto específico, como el Instituto Tecnológico de Morelia.
 
Al centrarse en estudiantes de ingeniería, el estudio se orienta hacia una categoría específica que tiene un valor significativo para los antecedentes de la organización. Este enfoque garantiza que las investigaciones sean más especializadas y específicas.
 
La investigación propuesta busca comprender las razones por la falta de herramientas tecnológicas y, por lo tanto, proporciona información real, verídica y crucial para diseñar estrategias efectivas para fomentar su adopción y mejorar el rendimiento académico en ciencias básicas y matemáticas.

\section{Bosquejo de Marco Teórico}

\subsection{Revisión de Literatura}

\begin{itemize}
  \item Estudio sobre las dificultades y problemáticas en el aprendizaje del cálculo diferencial.
  \item Investigación sobre el uso de herramientas web para el aprendizaje de cálculo diferencial.
  \item Experiencias con herramientas web de apoyo escolar en cálculo diferencial.
\end{itemize}
\subsection{Marco Metodológico}
\begin{itemize}
  \item \textbf{Teorías del aprendizaje:} Cognitivismo, constructivismo, aprendizaje significativo y metodologías ágiles en la enseñanza.
  \item \textbf{Teorías del aprendizaje mediado por tecnología:}  Teoría de la acción mediada, teoría de la actividad.
  \item  \textbf{Diseño instruccional:} Modelo ADDIE, modelo SAM.
\end{itemize}

\subsection{Marco Conceptual}

\subsubsection{Variables del Estudio}
\begin{itemize}
  \item \textbf{Variable Independiente:} Diseño de la herramienta web, contenido, actividades.
  \item \textbf{Variable Dependiente:} Desempeño Académico, satisfacción de los estudiantes.
\end{itemize}

\subsection{Marco Histórico}
La evolución del aprendizaje del calculo diferencial, los métodos tradicionales y los métodos con el uso de la tecnología.

\subsection{Marco Legal}
Normas que nos regulan en la enseñanza y las herramientas usables en las aulas del Instituto Tecnológico de Morelia. Las respectivas propiedades intelectuales, derechos de autor y propiedad digital del sistema.



\begin{thebibliography}{99}
  \bibitem{Vivar} Vivar, C. G., Arantzamendi, M., López-Dicastillo, O., y Gordo Luis, C. (2010). La Teoría Fundamentada como Metodología de Investigación Cualitativa en Enfermería. Index de Enfermería, 19(4). https://doi.org/10.4321/s1132-12962010000300011
  \bibitem{Brousseau} Brousseau, G. (2007). Iniciación al estudio de la teoría de las situaciones didácticas/ Introduction to study the theory of didactic situations: Didactico/ Didactic to Algebra Study. Libros del Zorzal.

\end{thebibliography}

\end{document}
